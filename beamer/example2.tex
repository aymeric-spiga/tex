\documentclass[pdftex,12pt]{beamer}		%% talk
%\documentclass[pdftex,12pt,handout]{beamer}	%% print

\usepackage{my_beamer}
%\usepackage{patch_french_beamer}
\usepackage{patch_english_beamer}
%\setbeameroption{show only notes}
%\setbeameroption{show notes}

\title[title]{TITLE}
\subtitle{subtitle}
\author[A. Spiga]{Aymeric SPIGA}
\institute[LMD / UPMC]{
Laboratoire de Météorologie Dynamique (Paris)\\
Université Pierre et Marie Curie (Paris)
}
\date[\today]{\today}

%\author[Aymeric SPIGA]{ \textcolor{blue}{Aymeric~Spiga}\inst{1,2} \and Fran\c cois~Forget\inst{1} \\ Stephen~R~Lewis\inst{2} \and David~P~Hinson\inst{3}}
%\institute[LMD]{\inst{1} \textcolor{blue}{Laboratoire de M\'et\'eorologie Dynamique, Universit\'e Pierre et Marie Curie [France]} \and \vskip-2mm \inst{2} Department of Physics and Astronomy, The Open University [UK] \and \vskip-2mm \inst{3} Carl Sagan Center, SETI Institute [USA]}

\begin{document}
%%%%%%%%%%%%%%%%%%%%%%%%%%%%%%%%%%%%%%%%%%%%%%%%%%%%%%%%%%%%%%%%%%%%%%
%%%%%%%%%%%%%%%%%%%%%%%%%%%%%%%%%%%%%%%%%%%%%%%%%%%%%%%%%%%%%%%%%%%%%%

	%%%
\slidetitle

	%%%
%\section{Objets examinés}

	%
%\slide{Planétologie comparée}{\image{0.75}{0.75}{decouverte/models/planeto.png}\source{Dowling, 2008}}
%%0.85 -- 0.75
	%
%\slide[c]{Stabilité atmosphérique (cas sec) : à retenir}{\vskip -0.5cm \doublecol{0.5}{\centers{Température}\image{0.95}{0.6}{decouverte/cours_dyn/stab_temp_legras.png}}{0.5}{\centers{Température potentielle}\image{0.95}{0.6}{decouverte/cours_dyn/stab_temppot_legras.png}}}
	
	%
%\slide[c]{Circulation générale, Terre : Saisons}{\vskip -0.4cm \doublecol{0.7}{\centers{Flux de masse (fonction de courant) \vskip -0.2cm \image{0.90}{0.65}{decouverte/cours_dyn/peixoto_stream.png} \sourcec[-0.7]{Peixoto and Oort, 1992}}}{0.3}{\centers{Flux radiatif net} \vskip -0.4cm \image{0.90}{0.65}{decouverte/cours_dyn/flux_net.jpg} \sourcec[-0.7]{Kandel and the ScaRaB team, 1998}}} \note{- Ascendances où il fait chaud (toujours pour la zone équatoriale)\\ - Les cellules constituent un mécanisme efficace pour redistribuer l'énergie excédentaire dans la bande tropicale vers la partie de la planète la plus déficitaire: l'hémisphère d'hiver\\ - Les cellules ne s'étendent pas jusqu'au pôles; le transport d'énergie vers les pôles est effectué par un autre mécanisme au-delà d'une certaine latitude (et même en sens inverse, cf. cellules de Ferrel)}

	%
%\slide{Bibliography}{\nocite{Spig:10bl}\nocite{Spig:09}\nocite{Spig:10comp}\scriptsize\bibliographystyle{apalike}\bibliography{/home/aymeric/Work/submitted/newfred}\normalsize}

%%%%%%%%%%%%%%%%%%%%%%%%%%%%%%%%%%%%%%%%%%%%%%%%%%%%%%%%%%%%%%%%%%%%%%
%%%%%%%%%%%%%%%%%%%%%%%%%%%%%%%%%%%%%%%%%%%%%%%%%%%%%%%%%%%%%%%%%%%%%%
\end{document}


